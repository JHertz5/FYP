\documentclass[11pt]{article}

\usepackage{times}
\usepackage{epsfig}
\usepackage{graphicx}
\usepackage{amsmath}
\usepackage{amssymb}
\usepackage{color, soul} 	%for highlighting
\usepackage{listings} 		%for writing code
\lstset{
basicstyle=\small\ttfamily,
columns=flexible,
breaklines=true
}
\usepackage[margin=1in]{geometry} %for margins

\setcounter{page}{1}
\begin{document}

\begin{titlepage}

\title{Space Brain - Implementing a Convolutional Network on an Embedded System with an FPGA for Space Applications Suitability Testing}
\author{Jukka (Jacobus) Hertzog}
\def\supervisor{Dr. Felix Winterstein}
\def\secondmarker{Dr. David Thomas}
\def\course{EE4T}
\def\cid{00828711}

\setlength{\parindent}{0pt}
\setlength{\parskip}{0pt}
\fontfamily{phv}\selectfont
{
\large
\raggedright
Imperial College London\\[17pt]
Department of Electrical and Electronic Engineering\\[17pt]
Final Year Project Interim Report 2017\\[17pt]
}
\rule{\columnwidth}{3pt}
\vfill
\centering
\makeatletter
\begin{tabular}{p{40mm}p{\dimexpr\columnwidth-40mm}}
Project Title: & \textbf{\@title} \\[12pt]
Student: & \textbf{\@author} \\[12pt]
CID: & \textbf{\cid} \\[12pt]
Course: & \textbf{\course} \\[12pt]
Project Supervisor: & \textbf{\supervisor} \\[12pt]
Second Marker: & \textbf{\secondmarker} \\
\end{tabular}
\end{titlepage}

\section{Introduction}
\label{sec:Introduction}

Convolutional Neural Networks (CNNs) are a powerful Deep Learning tool that can be used to solve extremely complex computational problems. In particular, they have gained popularity in computer vision and image classification applications, performing with very high accuracy. However, CNNs are extremely computationally heavy. As a result, implementing them on embedded systems, which typically have very limited resources, presents many challenges. A promising solution to this problem is the use of an FPGA, which provide a very high computational power with low power usage.

Consider a miniature research satellite (a CubeSat), that uses a camera to capture images to be classified. Normally the images would be transmitted to a ground station to be processed, but in this case, channel capacity during the transmission window is too limited and the raw data is too large for this to be feasible, so the idea is to process and classify images on board the satellite and just transmit the results. Therefore, the aim of this project is to implement a CNN on an FPGA-based system, and then to investigate the suitability of this system for space applications.

\section{Background}
\label{sec:Background}

\subsection{Convolutional Neural Networks}
\label{sec:ConvolutionalNeuralNetworks}

\subsection{CNNs Implemented on FPGAs}
\label{sec:CNNsImplementedOnFPGAs}

\subsection{FPGAs and Space Applications}
\label{sec:FPGAsAndSpaceApplications}

\section{Project Specification}
\label{sec:ProjectSpecification}

\subsection{Implementation of a CNN on an FPGA}
\label{sec:ImplementationOfACNNOnAnFPGA}

The first deliverable is the implementation of a CNN on an embedded platform. The hardware used is a product from Xilinx called a Zedboard, a development board for the Zynq-7000 System On a Chip (SoC). The Zynq-7000 consists of a Xilinx 28nm Artix-7 FPGA and a dual-core Cortex-A9 ARM processor integrated together. The processor, with it's easily developed code, will control the system, while the FPGA, being able to efficiently perform heavy computations, will be used to accelerate functions.

An important tool in this process will be Xilinx's SDSoc development environment. SDSoc is able to optimise and compile C, C++ or OpenCL source code for a Zynq system. The compiler generates software for the ARM core and, using a High Level Synthesis (HLS) tool, a bitstream for the FPGA. This allows the user to design the entire system, easily accelerating functions with the FPGA. SDSoc is a very powerful tool, facilitating the rapid development and design of firmware.

Another important tool will be Caffe (or Convolutional Architecture for Fast Feature Embedding). This is a deep learning framework that can be used to design, train, optimise and test CNNs, with a focus on computer vision\cite{jia2014caffe}. It also includes some examples of pre-trained CNN models that are ready for use. Using Caffe, it will be possible to generate C++ code for a CNN, which can then be implemented on the Zynq system using SDSoc.

\subsection{Investigation of Fault Tolerance in the Presence of Bit-stream Corruptions}
\label{sec:InvestigationOfFaultToleranceInThePresenceOfBitstreamCorruptions}

\section{ImplementationPlan}
\label{sec:ImplementationPlan}

\section{Evaluation Plan}
\label{sec:EvaluationPlan}


\bibliography{interimReport_ref}
\bibliographystyle{ieeetr}
\nocite{*}

\end{document}