\documentclass[11pt]{article}

\usepackage{xfrac}
\usepackage{times}
\usepackage{epsfig}
\usepackage{graphicx}
\usepackage{listings}
\usepackage{amsmath}
\usepackage{amssymb}
\usepackage{minibox}
\usepackage{color, soul}
\usepackage{float}
\usepackage{textcomp}
\usepackage[super]{nth}
\setlength{\intextsep}{5pt}
\usepackage[nodisplayskipstretch]{setspace}
\setstretch{1}
\usepackage{algorithm}
\usepackage{algorithmic}
\usepackage[utf8]{inputenc}
\lstset{
basicstyle=\small\ttfamily,
columns=flexible,
breaklines=true
}

\setcounter{page}{1}
\begin{document}

\begin{titlepage}

\title{Space Brain - Implementing a Convolutional Network on an Embedded System with an FPGA for Space Applications Suitability Testing}
\author{Jukka (Jacobus) Hertzog}
\def\supervisor{Dr. Felix Winterstein}
\def\secondmarker{Dr. David Thomas}
\def\course{EE4T}
\def\cid{00828711}

\setlength{\parindent}{0pt}
\setlength{\parskip}{0pt}
\fontfamily{phv}\selectfont
{
\large
\raggedright
Imperial College London\\[17pt]
Department of Electrical and Electronic Engineering\\[17pt]
Final Year Project Interim Report 2017\\[17pt]
}
\rule{\columnwidth}{3pt}
\vfill
\centering
\makeatletter
\begin{tabular}{p{40mm}p{\dimexpr\columnwidth-40mm}}
Project Title: & \textbf{\@title} \\[12pt]
Student: & \textbf{\@author} \\[12pt]
CID: & \textbf{\cid} \\[12pt]
Course: & \textbf{\course} \\[12pt]
Project Supervisor: & \textbf{\supervisor} \\[12pt]
Second Marker: & \textbf{\secondmarker} \\
\end{tabular}
\end{titlepage}

\section{Introduction}
\label{sec:Introduction}

Convolutional Neural Networks (CNNs) are a powerful Deep Learning tool that can be used to solve extremely complex computational problems. In particular, they have gained popularity in computer vision and image classification applications, performing with very high accuracy. However, CNNs are extremely computationally heavy. As a result, implementing them on embedded systems, which typically have very limited resources, presents many challenges. A promising solution to this problem is the use of an FPGA, which provide a very high computational power with low power usage.

Consider a miniature research satellite (a CubeSat), that uses a camera to capture images to be classified. Normally the images would be transmitted to a ground station to be processed, but in this case, channel capacity during the transmission window is too limited and the raw data is too large for this to be feasible, so the idea is to process and classify images on board the sattelite and just transmit the results. Therefore, the aim of this project is to investigate the suitability of using an FPGA-based system to implement a CNN in this CubeSat. 

\section{Project Specification}
\label{sec:ProjectSpecification}

\subsection{Implementation of a CNN on an FPGA}
\label{sec:ImplementationOfACNNOnAnFPGA}



\subsection{Investigation of Fault Tolerance in the Presence of Bitstream Corruptions}
\label{sec:InvestigationOfFaultToleranceInThePresenceOfBitstreamCorruptions}


\end{document}